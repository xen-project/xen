\chapter{Glossary of Terms}

\begin{description}

\item[Atropos] One of the CPU schedulers provided by Xen.  Atropos
  provides domains with absolute shares of the CPU, with timeliness
  guarantees and a mechanism for sharing out `slack time'.

\item[BVT] The BVT scheduler is used to give proportional fair shares
  of the CPU to domains.

\item[Exokernel] A minimal piece of privileged code, similar to a {\bf
    microkernel} but providing a more `hardware-like' interface to the
  tasks it manages.  This is similar to a paravirtualising VMM like
  {\bf Xen} but was designed as a new operating system structure,
  rather than specifically to run multiple conventional OSs.

\item[Domain] A domain is the execution context that contains a
  running {\bf virtual machine}.  The relationship between virtual
  machines and domains on Xen is similar to that between programs and
  processes in an operating system: a virtual machine is a persistent
  entity that resides on disk (somewhat like a program).  When it is
  loaded for execution, it runs in a domain.  Each domain has a {\bf
    domain ID}.

\item[Domain 0] The first domain to be started on a Xen machine.
  Domain 0 is responsible for managing the system.

\item[Domain ID] A unique identifier for a {\bf domain}, analogous to
  a process ID in an operating system.

\item[Full virtualisation] An approach to virtualisation which
  requires no modifications to the hosted operating system, providing
  the illusion of a complete system of real hardware devices.

\item[Hypervisor] An alternative term for {\bf VMM}, used because it
  means `beyond supervisor', since it is responsible for managing
  multiple `supervisor' kernels.

\item[Live migration] A technique for moving a running virtual machine
  to another physical host, without stopping it or the services
  running on it.

\item[Microkernel] A small base of code running at the highest
  hardware privilege level.  A microkernel is responsible for sharing
  CPU and memory (and sometimes other devices) between less privileged
  tasks running on the system.  This is similar to a VMM, particularly
  a {\bf paravirtualising} VMM but typically addressing a different
  problem space and providing different kind of interface.

\item[NetBSD/Xen] A port of NetBSD to the Xen architecture.

\item[Paravirtualisation] An approach to virtualisation which requires
  modifications to the operating system in order to run in a virtual
  machine.  Xen uses paravirtualisation but preserves binary
  compatibility for user space applications.

\item[Shadow pagetables] A technique for hiding the layout of machine
  memory from a virtual machine's operating system.  Used in some {\bf
  VMMs} to provide the illusion of contiguous physical memory, in
  Xen this is used during {\bf live migration}.

\item[Virtual Block Device] Persistant storage available to a virtual
  machine, providing the abstraction of an actual block storage device.
  {\bf VBD}s may be actual block devices, filesystem images, or
  remote/network storage.

\item[Virtual Machine] The environment in which a hosted operating
  system runs, providing the abstraction of a dedicated machine.  A
  virtual machine may be identical to the underlying hardware (as in
  {\bf full virtualisation}, or it may differ, as in {\bf
  paravirtualisation}).

\item[VMM] Virtual Machine Monitor - the software that allows multiple
  virtual machines to be multiplexed on a single physical machine.

\item[Xen] Xen is a paravirtualising virtual machine monitor,
  developed primarily by the Systems Research Group at the University
  of Cambridge Computer Laboratory.

\item[XenLinux] Official name for the port of the Linux kernel that
  runs on Xen.

\end{description}
