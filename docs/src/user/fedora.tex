\chapter{Installing Xen on Fedora~Core 4}

This section will help you in installing Xen 3 on Fedora Core 4 using various methods.

\section{Installing Xen from Source Package and binary package}

\subsection{Required Packages}
bridge\_utils


\subsection{Installing}

Download the source or binary tarballs available at \begin{quote} {\tt http://www.xensource.com/downloads } \end{quote}.

Extract the archive using following command:

\begin{verbatim}
tar -zxvf xen-*****-***.tgz
\end{verbatim}

cd into the xen directory.

To compile and install the source do

\begin{verbatim}
     make dist
     make install
\end{verbatim}


To install the binary tarball, all you need to do is run the \path{install.sh} script.

\begin{verbatim}
     #./install.sh
\end{verbatim}

\subsection{Installing Xen using yum}

To install xen, type the command

\begin{verbatim}
#yum install xen
\end{verbatim}

This will download the following rpms and install them:

\begin{itemize}
\item xen
\item bridge-utils
\item sysfsutils
\end{itemize}

Next we need to install kernel-xen0 and kernel-xenU. Type the command:

\begin{verbatim}
 yum install kernel-xen0 kernel-xenU 
\end{verbatim}

Note: This installs xen0 and xenU kernels and adds an entry in the grub configuration.
Getting Xen up and running

Once this finishes, you have xen0 and xenU kernels installed in the /boot filesystem. To boot into Dom0, edit the grub configuration file, which is menu.lst

Note: Installation using yum doesn't require the configuration of grub as mentioned below.

An example grub entry would be like:

{\small
\begin{verbatim}
title Xen Unstable(From Fedora Core 4)
          root (hd0,0)
          kernel /fedora/xen.gz dom0\_mem=230000 console=vga
          module /fedora/vmlinuz-2.6-xen0 root=/dev/Vol1/LV3 ro console=tty0
          module /fedora/initrd-2.6.11-1.1369\_FC4smp.img
\end{verbatim}
}

Also make sure that \path{/var/run/xenstored} and \path{/var/lib/xenstored} directories have been created. If they are not, manually create them.

Now reboot and select the xen0 option from the GRUB menu.

To check whether you are running the xen0 kernel, type \path{uname -r}

Now start the xend process:

\begin{verbatim}
xend start
\end{verbatim}

To check whether xend process is running or not, type the following command which lists the running domains.

\begin{verbatim}
#xm list
      Name              Id  Mem(MB)  CPU VCPU(s)  State  Time(s)
      Domain-0           0      219    0      1   r-----     28.9
\end{verbatim}

Since you haven't created any guest domains yet, you would see only Domain0.

Further Help and documentations

Besides the usual resources, see the Fedora Quickstart Guide \begin{quote} {\tt http://www.fedoraproject.org/wiki/FedoraXenQuickstart } \end{quote}
